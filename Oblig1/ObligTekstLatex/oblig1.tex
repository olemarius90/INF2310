\documentclass[11pt,a4paper]{article}
\usepackage[utf8]{inputenc}
%\usepackage[T1]{fontenc}
\usepackage[norsk]{babel}
\usepackage{textcomp,varioref,amsmath,listings,amssymb}
\usepackage[usenames,dvipsnames]{color}
\usepackage{graphicx}
\usepackage{epstopdf}
\usepackage{hyperref}

\lstset{
  frame=single,                   % adds a frame around the code
}

\title{Obligatorisk oppgave 1}
\author{INF2310 \\ Vår 2015}

\begin{document}
\maketitle{}

%\section*{Step A : Analyzing the textures}

\textbf{Dette oppgavesettet er på tre sider og består av én bildebehandlingsoppgave.}
\\
\\
Besvarelsen av denne og neste obligatoriske oppgave må være godkjent for at du skal
få anledning til å gå opp til endelig skriftlig eksamen i kurset.
Besvarelsene kan utarbeides i smågrupper på opptil to studenter, men det er ikke
noe i veien for å arbeide alene. Studenter i samme smågruppe kan levere identisk
besvarelse, men samarbeidet må framgå av navnene på forsiden av besvarelsen.\\
\\
Av side 1 skal det fremgå hvem som har utarbeidet besvarelsen.
\\
\\
Det forventes at arbeidet er et resultat av egen innsats.
Å utgi andres arbeid for sitt eget er uetisk og kan medføre sterke reaksjoner fra IFIs
side. Se \href{http://www.mn.uio.no/ifi/studier/admin/obliger/}{http://www.mn.uio.no/ifi/studier/admin/obliger/}.
Den skriftlige rapporten leveres primært som en PDF-fil som inneholder hele
besvarelsen, med figurer og bilder. Kode skal leveres i tillegg til PDF-filen.
Besvarelsen skal leveres via \href{http://devilry.ifi.uio.no}{http://devilry.ifi.uio.no}. 
Følgende er viktig:

\begin{itemize}
\item Alle filene må lastes opp hver gang man skal levere.
\item PDF-filen skal ha følgende navn: \textbf{inf2310-oblig1-brukernavn.pdf}.
\item Oppgaven skal kunne kjøres fra MATLAB- eller Python-scripts med navn:
\textbf{oppgave1.m} eller \textbf{oppgave1.py} (eventuelt \textbf{oppgave1a.m} osv.).
\item Spørsmål angående innlevering kan sendes til \\\textbf{magnuval@student.matnat.uio.no}.
\end{itemize}

Bildene det refereres til vil være å finne under:\\
\href{http://www.uio.no/studier/emner/matnat/ifi/INF2310/v15/undervisningsmateriale/bilder/}{http://www.uio.no/studier/emner/matnat/ifi/INF2310/v15/undervisningsmateriale/bilder/}
~
\\ 
~
\\
Oppgaven utleveres mandag 23. februar 2015.\\
Innleveringsfrist er mandag 16. mars 2015.

\newpage

\section*{Oppgave 1: Detektering av cellekjerner}
\textit{Viktig: Dere skal her programmere hele Cannys algoritme “fra bunnen av”, altså
ikke bruke andres implementasjon for verken Cannys algoritme eller noen av dets
bestanddeler, inkludert utvidelsen av inn-bildet, alle filtreringene, tynningen av
gradientmagnitudebildet og hysterese-tersklingen.}
\\

Du skal i denne oppgaven detektere kanten til cellekjernene i bildet \\\textbf{cellekjerner.png}.
Dette bildet er et lysmikroskopibilde av DNA-fargede cellekjerner i et tynt snitt av
svulsten til en prostatakreftpasient. Deteksjon av cellekjernene er nødvendig for å
kunne separat analysere cellekjernene for å f.eks. avgjøre aggressiviteten til
kreftsvulsten, noe som er viktig for å kunne bestemme hvor kraftig etterbehandling
pasienten bør bli utsatt for.
\subsection*{a)}
Programmèr en generell implementasjon av konvolusjon av et bilde med et
konvolusjonsfilter. Du kan anta at naboskapet er rektangulært med odde
lengder og har origo i senterpikselen. Ut-bildet skal ha samme størrelse som
inn-bildet og bilderandproblemet skal bli behandlet ved å utvide inn-bildet med
nærmeste pikselverdi.
\subsubsection*{Tips:}
Det er skrevet en testfunksjon som tester din implementasjon av konvolusjon mot MATLAB's implementasjon av funksjonen
\textbf{imfilter}. Testfunksjonen, samt beskrivelse av hvordan du bruker den, ligger på \\
\href{http://github.com/olemarius90/INF2310/tree/master/Oblig1}{http://github.com/olemarius90/INF2310/tree/master/Oblig1}.

\subsection*{b)} 
Implementèr Cannys algoritme. Bruk et ekte Gauss-filter slik som det er
definert på side 41 i notatet til den første filtreringsforelesningen. Altså, bruk formelen
til å regne ut verdiene, ikke en tilnærming slik som f.eks. på den påfølgende siden i samme
notat. Bruk den symmetriske 1D-operatoren til å estimere gradientene og benytt 8-tilkobling i
hysterese-tersklingen. For å tynne gradientmagnituden skal du benytte metoden
i notatet til den andre filtreringsforelesningen. Dette er også er beskrevet i
læreboka (DIP, Gonzales og Woods) på side 720-722. Kort fortalet er metoden for hysterese-tersknlingen
at man går igjennom hver piksel i gradientmagnitudebildet og setter verdien til 0 dersom minst én av 
de to 8-naboene i eller mot gradientretningen har større gradientmagnitude.

Siste del av Cannys algortime er definere to threshold \textbf{Th} og \textbf{Tl} for å 
analysere, finne og sette sammen kanter. Dette er, som resten av algortimen, beskrevet i 
forelesningsfoilene og i læreboka (DIP, Gonzales og Woods) på side ...

\subsubsection*{Tips:}
\begin{itemize}
\item[1.] For å tilpasse Gauss-filterets størrelse til parameteren \textit{sigma} kan du sette
hver lengde av filteret som én pluss 8 ganger \textit{sigma} rundet opp til
nærmeste heltall. I prinsippet er Gauss-filterets størrelse ubegrenset, men
verdiene til h(x,y) utenfor det angitte området er så liten at det ikke har
noen markant praktisk betydning å kutte ut disse.
\item[2.] Benytt implementasjon din fra a) til Gauss-filtrering og estimering av
gradientene.
\item[3.] Det er selvsagt tillatt å benytte andres implementasjoner for avrunding
og for beregning av kvadratrot og tangens invers, f.eks. MATLAB funksjonene
\textbf{ceil}, \textbf{sqrt} og \textbf{atan}.
\end{itemize}

\subsection*{c)}
Anvend din implementasjon av Cannys algoritme på bildet \textbf{cellekjerner.png}.
Eksperimenter med parameterne \textbf{sigma}, \textbf{Th} og \textbf{Tl} til du har funnet noen
verdier som gir et resultatbilde med flest mulig av kantene til cellekjernene og
færrest mulig andre kanter.
\\

Bildet \textbf{detekterte\_kanter.png} viser et mulig resultatbilde, 
men det er på ingen måte en fasit. Det er godt mulig du vil få et enda bedre resultat ved å bruke
andre parameterverdier!


\subsubsection*{Tips:} For å sammenligne forskjellige parameterverdier er det lurt å
normalisere gradientmagnituden slik at maksimalverdien er 1.
\\

Du kan benytte andres implementasjoner, f.eks. MATLAB-funksjonene \textbf{imread} og
\textbf{imwrote}, for å lese/skrive fra/til fil.

\subsection*{Hva skal leveres:}
\begin{itemize}
\item[I.] Tekstlig beskrivelse av implementeringene i a) og b).
\item[II.] Resultatbildet med tilhørende parameterverdier i c).
\item[III.] Drøfting av resultatbildet, inkludert hva som kjennetegner det metoden er bra
på og det metoden ikke er bra på, samt hvordan dette blir påvirket av å endre
parameterverdiene.
\item[IV.] Programkode.
\end{itemize}
Lykke til!
\end{document}
    
